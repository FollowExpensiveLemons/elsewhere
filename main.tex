%Todo: Carnegie roof, Nugent-Sharpe tunnel


\documentclass{article}
\usepackage{amsmath, amsfonts}
\usepackage{hyperref}
\usepackage[margin=1in]{geometry}
\usepackage{multicol}
\hypersetup{
    colorlinks,
    citecolor=black,
    filecolor=black,
    linkcolor=black,
    urlcolor=black
}
\title{Campus Exploration Manual (CEM)}
\author{RPI Elsewhere Club}
\date{Last Updated: March 2023}

\setcounter{secnumdepth}{0}
\begin{document}
  \maketitle
  \section{Preamble}

  Hello! This document has been compiled, updated, maintained and converted to latex like god intended by students just like you who are looking to expand their experience on campus. This guide should give you fairly detailed info on how to get most anywhere on campus. Ever wondered where to spend a few hours late on a weeknight? We’ve got you covered!

While the CEM tries to be comprehensive, please keep in mind that it is compiled by students! There is a lot to be said for doing your own research. Additionally, no matter what the CEM says, please always make sure to be prepared for any outing of any level! Take safety seriously, and remember that RPI administration has the final word on any disciplinary action.

Special Thanks To:
- Climber
- Thulu
- Axolotl
- Alcinius
- funfest
- Anyone else who adds! :D

  \pagebreak
  \section{Table of Contents}
  \tableofcontents
  \pagebreak
  \section{Introduction}
  So you’re wondering what the Freshman Tour\textsuperscript{TM} didn’t show you? Or perhaps you’re a senior looking to see a little bit of everything before you leave. No matter what brought you here, a desire for exploration ties us all together. Rest assured there is a lot to see on campus, and many places can be fun to return to many times!
  \subsection{Exploring Within RPI Guidelines}
  There’s a lot of fun to be had on campus, but we want you to have that fun safely! This means personal safety and academic safety. In the CEM we use our Patented 1-6 Guide\textsuperscript{TM} to stay safe out there. We rate safety on 2 different metrics, the first on a scale of 1-6 and the second A-F. Any ranking in the CEM will look like this: [1:A], with the first number being Physical Risk and the second being Administrative Risk.
  \subsection{Risk Rating Guide}
  \begin{multicols}{2}
  \noindent
  \subsubsection*{Technical Difficulty}
  \begin{enumerate}
    \item [(1)] Just walking; all doors unlocked
    \item [(2)] Simple maneuvers; climbing a ladder or shimming a lock
    \item [(3)] Sketchy maneuvers; climbing through a window or to head height
    \item [(4)] Difficult maneuvers; climbs that may require a partner, lock picking
    \item [(5)] Serious maneuvers; climbing skills required, high possibility of injury
    \item [(6)] Dangerous maneuvers; extreme stunts with a possibility of death

  \end{enumerate}
  \subsubsection*{Administrative Risk}
  \begin{enumerate}
    \item [(A)] 100 allowed at all times
    \item [(B)] Your presence here is reasonably justifiable
    \item [(C)] You’re gonna get a stern talking to
    \item [(D)] Disciplinary action is likely
    \item [(E)] Expulsion worthy
    \item [(F)] Extremely high risk of apprehension or potential for criminal punishment
  \end{enumerate}
  \end{multicols}
  

  When thinking about going somewhere, keep in mind this scale, but also what it implies. Odds are, if you don't think you should be somewhere, you probably shouldn't be there. Additionally, going somewhere at night often increases the administrative risk by 1, because most buildings lock after dark.
  

\subsection{Recommended Equipment}
Here are a few suggestions for things to bring on any excursion:
\begin{enumerate}
\item Flashlight, some places are dark.
\item Markers, for drawing in student art areas.
\item Camera. There’s some cool stuff around!
\item Water. Walking around is tiring.
\item Backpack. Snag some cool stuff from the tech dump.
\item Student ID. Student handbook requires you to have it at all times.
\item Eye protection and simple masks (both available from the Union) for dust protection.
Depending on the severity of possible exposure, other PPE like gloves can help
\item A screwdriver set: good for the tech dumps, invaluable door shims, etc.
\item Good shoes. Climbing stuff is hard in sandals.
\item Band-aids. It really sucks when you nick yourself on something.
\item Weather appropriate clothes, especially if you’re outside in the cold.
\item An old credit card for door shimming.
\item Lock picking tools (if you have them), but know that if you’re caught with these in your
possession, regardless of whether you’ve used them or not, it can be grounds for severe
disciplinary action.
\item Fridge magnets. cut them to size and put them over door jams to “prop” doors.
\item Rope for small climbs and safe descents.
  % \pagebreak
\end{enumerate}
  \subsection{General Tips}
  Here are some tips from veteran explorers on having a good time while not doing anything frowned-upon:
  \begin{enumerate}
    \item The best way to avoid trouble is only entering spaces through unlocked doors that have no posted warnings or signage. This way, even if confronted, you can plead ignorance and all they can do is tell you to leave and not return.
    \item The DCC, JEC, CII and J-Rowl are all connected by tunnels. The interior doors that connect the tunnels are always unlocked, and there is always one exterior door on one of those buildings that is unlocked. Seriously, there is always at least one door unlocked. The tunnels are your friend for exploring after hours!
    \item The more hypothetically dangerous a place is, the more you hypothetically shouldn’t be there. Admin doesn’t want students getting hurt, so if you could get hurt somewhere, they don’t want you there!
    \item Don’t get caught with a marker or paint in your hand.
    \item If confronted by a public safety officer, remember that they likely don’t want to be there anymore than you do. Be polite, tell the truth about what you’re doing and they will let you slide 9 times out of 1\item Especially if a door was unlocked, and you show them the door, they can ask you to leave, nothing more. Don’t run from public safety!
\end{enumerate}

      \subsection{Tips for Risky Exploration}
If you have made the decision, fully understanding the risks, to explore things on campus that are less approved of, there are some things you should know. First, anything provided in this guide does not attempt to guarantee your safety should you chose to do something possibly dangerous. Second, the CEM does not condone the willful breaking of the policies laid out in the RPI Student Handbook.

All of this in mind, there is much to be said for setting your sights a little higher in exploration. Many of the best views on campus can only be obtained through some climbing. Generally this kind of exploration is looked down on by Administration. There are a few things you can do on campus that, if caught, will result in your immediate expulsion:
\subsubsection{Grounds for Expulsion}
\begin{enumerate}
\item Enter the steam tunnels, at any time, for any length of time. These are not to be confused with the tunnels under the DCC complex, those are fine! Getting caught in these tunnels under any pretenses will result in you finding a different school. See Appendix D for more steam tunnel info.
\item Get atop the JEC or CII roofs. RPI does not want the negative publicity from a kid dying falling from one of these roofs. Moreover, both of these roofs have fume hoods which spew all kinds of dangerous stuff into the sky. It’s dangerous for you, and the school doesn’t like it.
\item By the same token, any roof with a fume hood is more likely to result in more serious disciplinary action.
\item Disabling or destroying security equipment used to protect buildings. If you open a window to hop in a building, that is one thing. But if you destroy a card scanner, a security camera, or anything of the sort, it is grounds for both expulsion and a hefty fine.
\end{enumerate}

Now that you understand what to absolutely not do, here are some tips (in no particular order) for staying safe out there:
\subsubsection{Staying Safe}
\begin{enumerate}
\item Allied Universal, the company that is contracted to patrol a lot of the off campus housing, as well as pub safe, rotates its nightly shift at 10 pm and 6 am. During the winter it is still dark at 6am, and it should always be dark at 10, and you’ve got at least 5, sometimes 10 minutes during which patrols aren’t running at all.
\item Pubsafe goes through and locks many of the campus buildings, starting with all the outer doors of Low CII at 10pm, continuing to the JEC and Greene, and then to Sage and Ricketts.
\item Card readers on campus log scans, so if you beep into a building, public safety can check that. Use this to your advantage if you beep into a building before it gets late, and stay there for a bit, because that can be easily proven.
\item Your level of organization should be directly proportional to the potential risk of your task. When in doubt, It’s better to be over prepared. Consider having a lookout, someone who is willing to hang back to keep visibility on your area.
\item If you make a plan and something goes wrong, turn around. There are plenty of nights in the semester, and it’s better to slip away than to try and make do.
\item The security center in the Public Safety building is not massive, and it is not heavily staffed. Almost every camera on campus is not monitored. The notable exception to this are the cameras in the union, which are monitored by students most of the time. However, camera data is stored for some time, so don’t go breaking things!
\item Despite widely speculated rumors about security on campus, the fact of the matter is that most of the campus is less secure than it is made out to be. Public Safety has a few cameras up, but there are no motion detectors, door alarms or other systems that function universally across campus. This isn’t to be said that such security systems exist on a building by building basis, but tripping these sets off a building specific system, which must be responded to. Noteable alarmed areas include:
  \begin{enumerate}
    \item The Observatory on J-Rowl 
    \item The Sage Boiler Room
    \item Library Stairwells
  \end{enumerate}
\item There are some buildings that are just absolutely not worth your time. They are too locked down, and the risk of disciplinary action is too high. These buildings are:
  \begin{enumerate}
    \item Troy Building (It’s Shirley’s building for god’s sake)
    \item CBIS (It’s shiny, new and well secured)
    \item VCC (Millions of dollars worth of equipment and data)
  \end{enumerate}
\item You can get many places if you take the attitude that you’re willing to break things to enter a location. Please don’t do this. It is unsustainable for the future of other explorers on campus (don’t ruin it for others!) and it will get Public Safety to pull Troy PD onboard to charge you with crimes instead of slapping you on the wrist.
\end{enumerate}
Good luck out there explorers!

\section{Trips by Building}
This is the real meat of the CEM. An alphabetized list of trips per building, rated on our 1-6 scale (see above if you didn’t read about this). Generally, trips will be listed from lowest to highest risk. Any trip with a risk of [1:A] can be considered a good walking trip for anytime. Other resources, such as building maps, pictures, and anecdotes can be obtained from the Elsewhere Club Discord.



\pagebreak
\subsection{'87 Gym}
\subsubsection{The Entire Building [1:A]}
Rarely in the CEM do we advise an entire building as an exploration, but the ‘87 Gym is unique in this respect. It’s a narrow, labyrinthine structure built before the ADA was in the building codes. Its entirely possible to lose yourself in there and wander around for a bit. Special recommendation: the old biking track, now for running. The building is covered in security cameras, inside and out, so make sure to do a spot check before going after hours or into a locked room.
\subsubsection{‘87 Gym Pool [2:D]}
The ‘87 Pool was closed relatively recently, and only locked up in 2009. Visiting is a once in a career experience, and rewards the intrepid explorer with a place few people ever see. The easiest door to shim is an unmarked door by the north (sage street) entrance. When looking at the large double doors, the door is on the left at the end of a small hallway. To get down to the main floor, follow the rampart to the opposite side of the room and either climb down to the viewing section or walk across the wooden plank. Make sure to sign the scorer’s table, check out the pool floor, and make note of the old swimming record board on the wall. The room was closed due to asbestos in the pool insulation, so it is highly recommended that you bring a respirator.
\subsubsection{Secret Apartment (Coach’s Annex) [2:D]}
Located in the Northeast corner of the building is a 2-floor 3-bedroom apartment, complete with a kitchen, bathroom, and spacious closets. The easiest way to enter is from the pool through the eastern double doors at the top of the bleachers. From there go up the stairs to the left. There is also a door leading to the apartment from the main gym stairs on the east side of the building.
\subsubsection{‘87 Gym Roof [4:E]}
This roof is a technical, tricky ascent. Unlike most roof ascents, which are done mainly from indoors, this roof has to be taken from the outside. While it is possible with a team working together, a ladder makes the first hop much easier. Begin the ascent from the South-West corner, get up the first 10 foot block or so, then scramble up the rest. The first part is the hardest, and the easiest one to get injured on. Take special care coming down. Much like Greene, the top of the ‘87 Gym is roofed with slick, steep metal sheeting, except the sheeting on the Gym is covered in ice-slide spikes.


\pagebreak
\subsection{Academy Hall}
\subsubsection{Basement Rooms [2:B]}
Go down the stairs on the Northwest side of the building to reach the 1000 level of the building. From there proceed south past the large gym room on the right to a gray metal door leading to a storage area. At the end of the storage room is another door. Through there is a suite of three vacant rooms and some ancient stairs that now, due to renovations on the building, currently lead to the ceiling.



\pagebreak
\subsection{Amos Eaton}
\subsubsection{Edwin Brown Allen Room [1:A]}
On the 3rd floor, room 315 houses a lounge with a ton of old books and locked glass cases full of cool artifacts. Great study space during finals.



\pagebreak
\subsection{Armory/Mueller Center}
\subsubsection{Armory Lower Roof [3:C]}
A nice easy climb from outside the student auto shop up some old debris will bring you up to the lowest roof of the Armory. A nice place to visit once, but incredibly visible to everyone around and right next to where Public Safety patrols depart. Since it’s so low and easy to get to though, you’re likely not to get chewed out for being up there.
\subsubsection{Armory Pool Roof [5:C]}
A seemingly easy jaunt up from the Lower Roof, this slanted metal roof is incredibly slick. It’s barely worth the marginally higher view from up there, and the chance of sliding down the roof like you’re on some maniac designed slide is too high to ignore.
\pagebreak

\subsection{BARH}
\subsubsection{BARH Roof [4:D]}
A fairly trivial climb, being perched atop BARH offers you fantastic view of campus. You are, however, visible from all 4 directions, so be careful to hunker down as pub safe does their rounds. Walk around to the back of the building, where you will find a fence surrounding the HVAC unit. Climb this fence and use the wall to climb atop the ledge above the doorway to the dining hall. The rest of the levels on the roof are all able to be climbed solo.

\pagebreak
\subsection{Carnegie Hall}
- Carnegie hall is unlocked during the day, but at night, both doors are locked. If you feel like taking the [5:C] risk, you can get in after hours. Go around the back where the emergency exit is. To the left of it, there is a window that leads to the women's restroom. Shimmy your way up to it, push the window open, and climb under it. If you fall, you’re facing a 10 foot drop onto a bed of rocks, so be careful.

\subsubsection{Carnegie Attic [1:B]}
Although the door to the old Carnegie drawing room at the top of the stairs is often locked, the reward for getting in is top notch. The deadbolt is missing, so you can just walk right in. This attic is one of the spookiest places on campus during a storm. Turn off the lights and have a good time!
\subsubsection{Carnegie Basement [1:B]}
Accessed through the main stairwell in Carnegie, or via the North basement door which is occasionally propped. Most rooms are unlocked or simply have no doors on them. Various chalkboards will appreciate some additions to their collection. There are two halves to the basement, the west half has furnished rooms and offices, the east half has various derelict mechanical rooms. There are also some laboratory rooms in the middle. We recommend that you have your first run through at night - it is a surreal experience.
\subsubsection{Carnegie Basement Mechanical Rooms [1:C]}
There are a few locked rooms in the basement. They have very interesting old plumbing and equipment inside, worth checking out if you like old tech. The East side of the basement has a locked gate that is suspected to be alarmed, protecting steam regulation equipment. The North side of the basement has a walled-off entrance to an unconfirmed steam conduit that leads to Walker Laboratory.
\subsubsection{Carnegie Roof [?:?]}
%TODO

\pagebreak
\subsection{CII}
\subsubsection{CII Tech Dump [1:A]}
You can get to these tech dumps from the North-East corner of the CII. Enter and follow the East hallway midway through the building and then take a right. Work your way back and you’ll find the customary cardboard bins. While generally a little out of the way, this tech dump usually has computing equipment. This is also the best place on campus to recycle batteries if you need to!
 \subsubsection{The Throne Room [1:C]}
Also known affectionately as “Floor 9 and 3⁄4” this is the highest point you can attain on campus without bending any rules. As a result, it also used to be the campus’ foremost anonymous student art space. Unfortunately, it gained quite a bit of attention due to its accessibility and unique location, and recently, the administration has taken serious measures to suppress attempts at graffiting the area.
- Update - Fall 2017: RPI now appears to be more actively pursuing disciplinary actions against students who are caught adding art to the wall here, claiming that recent art has been “offensive” and reflecting poorly on the school when business associates come to the CII. If you go up here, be careful.
- Update - Fall 2019: RPI has installed a security camera on the 9th floor, facing into the stairwell window. Graffiti is being actively painted over. It’s advised to only go up here for a quick visit, at least until the heat dies down.
- Update - Spring 2020: The camera is now gone, but no new art has been posted. Still considered a risky place to draw on for the time being.
- Update - Spring 2020: A padlocked gate has been constructed in the section of the stairwell leading to the Throne Room. It’s not too difficult to get around, but the area is now considered a much riskier place to explore. Do so at your own risk!
- Update - Fall 2020: Gate has been removed. New artwork has been spotted. Still, take care while visiting, as pubsafe has a lot less to do right now.
\subsubsection{CII Lower Roofs [3:D]}
Currently only possible with a team and a rope ladder, the three lower roofs of CII are fairly easy to navigate by maintenance ladders once reached. From the South-East CII corner of the CII, find a window that’s unlocked and toss your ladder down. Descend carefully (don’t break windows!) to the lower roof and follow maintenance ladders around the rest of the roof. Somebody has to stay inside if you attempt this, or until a ground route is found.
A window on the sixth floor of the CII opens onto a ledge leading to the roof. As of Nov. 2018, this window is locked. Do your best to close the window after climbing through it: it is easy to open from the outside. Once on the ledge, climb the ladder to the lower roof. Be quick, as you will be very exposed from this location.

 \subsubsection{CII Roof [4:E]}
The CII Roof is currently only accessible by keyed entry, or a particularly skilled locksmith. The door that opens from The Throne Room onto the roof is locked by the master key to most school facilities. It is possible to create a facsimile of this key with a key blank, some on-brand bump keys and enough other keys to run averages. Of course, if you’ve put in that level of work, chances are you aren’t reading this. Once you’re on the roof, stay on designated walkways to avoid fume-hood blowoff. Most of the edges are walled by a waist height extension of the facade, and you’re high enough to avoid attention from basically anyone.
- Update - Fall 2016: The external doors of the south tower on the roof are locked.
\subsubsection{CII Elevator Room [5:E]}
Perhaps the most infamous location on campus aside from the steam tunnels, the CII elevator room has been a subject of contention for many years. There is an “elevator surfing” guide readily accessible online which suggests a manner of accessing the room. Follow the guide at your own risk. Public Safety is not notified immediately should an elevator stop, but the elevators do not resume motion once their interior doors open. Stopping an elevator with the required tools to exit between floors looks mighty suspicious when Troy Fire comes to rescue you.
- Update - Summer 2017: The machine room you can access via this method is protected by a padlocked grate, and the door to access the roof is locked.



\pagebreak
\subsection{Cogswell Labs / Materials Research Center}
\subsubsection{Cogswell Roof [5:E]}
The best way to reach this roof is from the parking garage next door. You can hop onto the large concrete wall around the Cogswell HVAC systems from any side basically, or by climbing up and HVAC vent on the parking garage. You can either follow the concrete wall to the Cogswell roof itself, or drop down into the HVAC area. Fair warning given, it is deceptively hard to get back out of the HVAC pit. On the Cogswell roof, an amazing view of Troy can be had from the West end.
\subsubsection{MRC Basement [1:B]}
Entering the Materials Research Center from the main entrance, immediately take a right, and a left, heading toward the back of the building. Turn down the hall and on the left you will see the MRC tech dump. Across from the tech dump is a door leading to the basement. In the basement is the main supplies room for the MRC, home to large (mostly locked) cages of defunct research instruments, a solvent container dump, and even some radioactive material!


\pagebreak
\subsection{Commons Dining}
\subsubsection{Commons Third Floor [1:B]}
Right as you exit from the dining hall onto the stairs down to the mailroom, there are two doors on your left. The second door leads to the (employees only) third floor of Commons, which isn’t signed at all and is kind of neat to look at. There are mainly storage rooms and some Sodexo offices.
- Update - September 2019: There is now a sign posted on the door leading to the third floor stating that if you’re caught up there, PubSafe will be called.
\subsubsection{Commons Roof [4:D]}
Accessible by a fairly simple climb up the East side of the building, this little climb has been attempted by leagues of freshmen, as well as a few adventurous upperclassmen with a desire to take an ice cream machine. The climb is fun, and you can wave to your class members in their dorms. The skylights are alarmed and there are a lot of people around, so get up, get your selfie and get down.


\pagebreak
\subsection{DCC}
\subsubsection{DCC Tech Dump [1:A]}
On the South end of the building, the tech dump is fairly easy to find. Follow the second floor DCC hallway that enters all the big lecture halls until it dead ends. This tech dump is generally a toss-up content wise, but always worth a visit.
\subsubsection{DCC Tunnel [1:A]}
Under the DCC, through a pair of glass doors next to WRPI lies the tunnel that connects the DCC to the JEC. The walls are almost always covered in student art of all kinds, and the amount of pipes, fixtures and other things on the walls means that the entire hallway is unlikely to ever be completely redone. Bring a few friends, some markers or paint and leave your artistic mark. Also be sure to clap or whistle at one end of the hallway, the echo down the all concrete passage is wild.

 \subsubsection{DCC Roof [4:D]}
Go under the concrete stairs around the back of the dcc and walk to the corner with the pipes. climb up the vertical pipe and into the narrow passageway. Walk until you find the ladder on the left side. make sure there is no one in the janitor closet with the window that faces this ladder. Once you ascend the ladder turn right and proceed up to the roof. its really big and really flat. the further away from the edge you are the harder it is for anyone to notice you.
\subsubsection{DCC Graffiti Cave [4:D]}
Once you are on the DCC roof, if you walk south along the satellite dish platform, you will get to a 15 foot ladder going down. When you climb down the ladder, you will be on the other side of the chain link fence visible from near the concrete stairs. Walk north and pop a squat and walk under these 3 metal box machine things(Unsure what they are). You will then encounter an AC unit looking box, and a 9 foot wall. Climb the 9 foot wall. There will be a grid of wood planks and metal wire and a 2 foot by 4 foot gap near the left hand side of the wall top. You’ll see 5 paracord wires, which you use to lower yourself onto the floor. Keep walking, and eventually you’ll make it to the graffiti cave. Bring a flashlight to see myriad tags.
\subsubsection{DCC Catwalks [4:C]}
On the roof of DCC-308 and DCC-318 is a catwalk that leads to a small room nestled between the two lecture halls. It can be accessed by stacking a chair onto the table to the right of DCC-308, and then pulling yourself onto the catwalk. In the room, there is an exit that can be used to let friends in. Be careful!

\pagebreak
\subsection{ECAV}
\subsubsection{Harkness Box Roof [3:B]}
The stands to the West of Harkness Field have an announcer box on top that you can climb onto fairly easily. It’s a fun place to hang out, and if you sit down it’s likely no one will notice you. Watching intramural sports from up there is a good evening hobby, especially if you have a friend playing.

\pagebreak
\subsection{EMPAC}
\subsubsection{EMPAC Lower Roof [3:D]}
Despite widely speculated rumors that there are lasers keeping the roof of EMPAC secure, it is likely that the only secured parts of the roof are the few hatches and doors. The climb to the roof has to be done completely from the exterior, and is easiest from the South-East side. It is stupid easy with a ladder. The view of Troy from the EMPAC roof is fantastic, and it is an absolutely fantastic place to see the sunset from. If you do bring a ladder with you, for God’s sake, please bring the ladder up onto the roof with you.
\subsubsection{EMPAC Upper Roof [3:E]}
While not particularly challenging to get onto from the lower roof, especially with a ladder, this is not really a great place to be. The surface of the upper roof of EMPAC is not designed to be walked on, and its curved enough to make the edges legitimately dangerous. The view isn’t much better than on the lower roof, unless you want to see the North side of campus.

\pagebreak
 \subsection{Greene Building}
\subsubsection{Green Roof [5:D]}
This neat, central campus view has become much harder to attain since the South fire escape of Greene was removed in 2016. Fortunately, there are almost always archies in the building. From the top of the central stairwell in Greene, enter the South studio and open one of the small, horizontal windows along the top of the wall. Climb out and onto a fairly spacious ledge. It is possible to climb farther up the roof, but it is very slippery and steep, and sliding down it will toss you off the edge and down 5 stories.
\pagebreak
\subsection{J Building}
\subsubsection{J Basement [1:B]}
Enter the J Building through the J-lot entrance that leads to the long skybridge to the 3rd floor. Go across the bridge and into the main lobby. Take a left toward the south and head down the hall to the elevator. Take the elevator to the basement and immediately take a right. There is an empty room on the southeast corner, and a mechanical equipment room with some interesting machinery.
\pagebreak
\subsection{JEC}
The JEC is filled mainly with labs and lab classrooms for engineering students. As such, most of the upper floors are fairly mundane, while the basement has some neat stuff in it!
\subsubsection{JEC Tech Dump [1:A]}
The JEC tech dump routinely has the coolest stuff out of all of the tech dumps. As the dumping point for a lot of old engineering equipment, office tech and used experimental setups, it’s worth a look every few weeks if any of that sounds interesting to you. You can also climb the ladder up to an HVAC filter storage area for a great view of the shop there!
\subsubsection{JEC Tunnels [1:A]}
The JEC is the “hub” for the tunnels that run under the campus. From the JEC’s basement you can get to both the DCC and J-Rowl. See the respective buildings for entries on each tunnel.

 \subsubsection{JEC Student Lab Roof [2:D]}
Accessed from 4th floor windows on the South side of the JEC, this small roof sits over the IED shop as well as IED classrooms. The roof has lots of cover (mainly HVAC vents) but you’ll be visible to anyone looking down at you from surrounding windows. Fairly easy to access, and a nice spot to sit down and chill on an evening without being bothered.
\subsubsection{JEC Machine Room [2:D]}
At the lower end of the JEC-DCC tunnel is a door leading to the central machine room for the engineering center. This large basement room contains the boilers, switchboards, electrical and HVAC systems for the whole building. From the entrance door, turn left out the door to the outside air intake area. Climbing up the metal ladder leads to an unlocked hatch on a grate that pops out right behind Mr. Jonsson himself.
\pagebreak
\subsection{J-Rowl}
\subsubsection{J-Rowl Tunnel [1:A]}
The tunnel connecting the JEC to J-Rowl is certainly the path less traveled. It’s a bit harder to find, and has less in it. Bring some friends and some markers and spice the place up! Low traffic means that you’re unlikely to be interrupted. There’s also a sketchy elevator that is even more isolated, making it better for more involved graffiti projects.
Update - Spring 2020: The elevator is currently down for maintenance.
\subsubsection{J-Rowl Mini Tech Dump [1:A]}
To get from the J-Rowl tunnel to the main building, you’ll come up into what appears to be a broom closet. It’s really the main dumping point for old school of sciences tech that they can’t take all the way to the JEC. On occasion there’s some neat stuff!
\subsubsection{J-Rowl Machine Room [1:C]}
Accessible via 3 entrances on the 4th floor, this room is usually easy to get into, and often unlocked. One of the doors can be shimmed as well. This large room has all of the non-fume hood HVAC machines that can’t go on the roof, as well as a variety of other bits and bobs, including some WRPI equipment. In the very back of the room, you’ll find a wooden chair facing a wall. Student legend says that no one has ever lasted more than 5 minutes sitting in the chair in complete darkness, with a jumble of machines running behind them. Bring friends, and see how they fare!

 \subsubsection{J-Rowl Elevator Room [2:C]}
Sometimes left unlocked, this room houses the upper machinery for J-Rowl’s elevator, which has been in service longer than everyone reading this has been alive. Always a good door to check while on the 4th floor.
\subsubsection{J-Rowl Roof [3:D]}
Accessed via East facing windows on the 4th floor, the J-Rowl roof is a nice central spot on campus to get a view. With the battleship like grad-student offices above you and the observatory in front, there are plenty of cool things to see. Almost the entire roof has a fence around it for safety, and please walk on the designated walkways! The West side of the roof has lab fume hoods, so visit at your own (administrative and physical) risk. If you want to come here without administrative risk,you can come to public observing Saturday nights 8-10 PM and spy on people in the library through 4-foot long binoculars.
\subsubsection{Control Room [1:C]}
An unmarked door on the 4th floor near the main stairs opens to a stairwell leading up to the J-Rowl Penthouse level, also called the control room. Four slanted glass walls feature 360° views of the south campus. A few grad student desks and a couch furnish the room, with stairs leading up to a padlocked roof hatch.
\pagebreak
\subsection{Ricketts}
\subsubsection{Ricketts Attic [1:B]}
This attic is accessible via two stairwells on the 4th floor. One is to the right of the main stairwell right before a tiny restroom, the other is to the left of the main stairwell through a suspicious atrium. There are various locked and unlocked storage cages, not much to see. But it is an interesting place to be, few students travel through the attic.
\subsubsection{Old AltLounge [1:B] (R.I.P.)}
Named after Prof. Elmar Altwicker, who used the room up until its abandonment in the early 2000s, this room is located in the attic of the Ricketts building, room 504. This room was used by the RPI S.A.S.S. club as an open lounge up until Spring 2018. Now all of the furnishings have been removed, but you may still find some traces that students have left behind.
- Update - Fall 2019: Area has been completely locked up by the administration.

 \subsubsection{Supersonic Wind Tunnel [1:B]}
This disused wind tunnel occupies a large open space on the 1st floor of Ricketts, room 101D. Turn left in the main atrium, there will be a door on the first right that is almost always unlocked. Follow the long tunnel until the end.
- Update - Fall 2018: this room has been cleared out and is being prepped for renovation.
\pagebreak
\subsection{Sage Labs}
\subsubsection{Sage Annex Attic [2:C]}
Also known as the Archaeology Attic, this neat little space can be accessed by climbing a ladder in a room in the far North-East corner of the Sage Annex. If you know where the Archaeology Grad Lounge is, the ladder is in the adjacent classroom. The doors to this area of the building are always unlocked, so if you’re in the building, you can see the attic. Light switch is on the floor at the top of the ladder. Bring a flashlight!
\subsubsection{Sage West Attic [2:C]}
This attic is accessed at the very top of the West stairwell of the Sage Main Building. The door is occasionally unlocked, and is easily shimmed. Back when Sage was built, the massive tank in this attic held water, and the gravity fed water pressure to the labs below. You can also see where the furnace heated water. This is a great visit at any time of day, with fantastic lighting during the day and spooky darkness at night. Tread carefully on established walkways or you will fall through the ceiling. The floor is also very thin, be quiet, especially during the day, or everyone below will hear you moving around.
\subsubsection{Sage East Attic [2:C]}
A rarely explored treat. In the east hallway of the 5th floor, look for a door on the south side of the building with a placard that reads “Attic.” Door is almost always locked, so lock must be shimmed or picked. Up the stairs is a large open attic space divided into two sections. One with rows of wooden shelves, an ancient HVAC system and a wooden workbench. The other has a ladder to a locked roof hatch, a large copper mesh cage, a lone urinal and a sink. Not recommended during the day as the floor is exceptionally creaky.

 \subsubsection{Sage Fire Escape [6:E]}
One of the more challenging routes, if you attempt it you’ll be rewarded by being somewhere nearly entirely unused. The exterior fire escape of Sage on the North side of the building is a newer addition to the building, inaccessible from most normal routes. To get there, exit through a window onto the tallest of the middle roofs of Sage, the one with the pyramidal skylight. Do a ledge walk around the building, and a hand-by-hand across an I-beam to swing onto the stairs. If you fall and survive, be prepared for a hefty medical bill and an expulsion letter.
\subsubsection{Towing Tank Laboratory [3:B]}
On the first floor (basement) of sage, near the elevator is a door marked “Towing Tank Laboratory.” If the door is unlocked, or you have a skilled burglar in your midst, you’ve got to check it out. After stepping over a bunch of cleaning supplies and janitor tools, you’ll see some wooden stairs leading up to a large pile of junk. Past that is a narrow 100-foot long room with a 3-foot deep channel with steel walls along the length of the room. It is in disrepair and has plenty of junk tossed in it, but if you walk along the length of the catwalk on the side, you can see some of the old equipment used to test various fluid properties in the tank.
\subsubsection{Sage front grate thing [2:B]}
The windows in the bathrooms of Sage Labs second floor are trivial to open (if they are not already open). Through the window is access to the space underneath the grates by the front door. Interesting and picturesque. If anyone asks why you are there, say you dropped something down the grate.
\pagebreak
\subsection{Sage Dining}
\subsubsection{Sage Lower Roof [3:C]}
From the North-West side of Sage Dining, it is a relatively easy climb onto the roof above the dining area of Sage. This climb is, however, in full view of the entirety of Quad. Like most climbs with significant visibility, get up, take your pictures and get down.

As of 2023 it seems that this roof is easily accessible. Both of the doors are kept unlocked to the roof and no climbing is necessary.
\pagebreak
\subsection{Union}
\subsubsection{Union Tech Dump [1:A]}
Ride the Union elevator down to the basement and take a right to be greeted by one of the most prolific tech dumps on campus. Almost all the tech equipment that gets recycled by any of the hundreds of campus clubs gets put into this tech dump. Additionally, easy 24 hour card access to the Union makes this tech dump a great spot to check.

 \subsubsection{Electrical Distribution System [1:E]}
Accessed from a few sets of metal double doors on the North side of the union basement. These doors are occasionally unlocked, and can be opened with excessive force. It houses most of the distribution equipment for the Union and surrounding buildings. This equipment is high-voltage and will kill you if you are stupid. So don’t be.
\pagebreak
\subsection{VCC}
\subsubsection{VCC Tech Dump [1:A]}
It would stand to reason that the tech dump in the computing building would have consistently the coolest things in it. However, unless you’re looking for materials for an art project, this tech dump is not a good place to find functioning equipment. Couple this with the VCC’s odd hours and this tech dump is probably one of the least visited on campus.



\pagebreak
\subsection{Walker Laboratory}
\subsubsection{Walker Sub-Basement [2:C]}
From the east entrance across from Sage Labs, walk down the stairs to the basement stockroom. With the stock room front window on your right, ahead is a door marked “stair.” Continue down the stairs to the sub-basement. On the right is a door that can be easily shimmed leading to a small equipment room and a black gate leading to a large open spooky cellar with some old chairs.



\pagebreak
\subsection{West Hall}
\subsubsection{West Tech Dump [1:A]}
The West tech dump sits on the lowest floor of West, look for the distinctive cardboard bin. This tech dump often has old or disused AV peripherals, cables and occasionally monitors and other electronics. West Hall is harder to get into at night, so as far as nightly tech dump runs go, it is out of the way. The chance for unique finds definitely necessitates occasional visits though.

\subsubsection{West Basement [1:A]}
West hall has a long and illustrious history, complete with all of the horror movie tropes. The basement of this hallowed hall of our Institution has been in use by generations upon generations of RPI students since time beyond memory. Check out the old gym if its unlocked, and a multitude of other empty and abandoned rooms. A great place to spend some time at night, but keep your holy water handy.
\subsubsection{West True Basement [3:C]}
The floor that most know as the Basement is in fact only the Ground Floor. West Hall has a dungeon below the expansion that is best accessed through the outside. On the north or south side of the lowest part of the building, there are some square-shaped windows. They can be pushed on the bottom if they are unlocked, and will swing inwards. Climb in, being careful to avoid stepping in the rat traps. Enjoy rooms full of dust, maintenance supplies, and derelict steam equipment. To exit the basement, head up a narrow stairwell on the southeast side, then out into the Ground Floor.

- Update March 2023: There is an alternate entrance to the West Hall True Basement that is accessible from the inside. Find the elevator on the ground floor then walk two doors down (towards the entrance). You will find a room full of old desks and cabinets and stuff. In the back-left of the room there will be a door with a hole above it. Crawl through and you will exit right at the narrow staircase which will lead you down to the True Basement.

\subsubsection{West Lower Roof [3:D]}
From the top floor of West, climb out any of the windows onto the central roof of West Hall. The windows open onto a ledge which you can easily navigate to the other West Hall roofs. In the center of the building is a recessed section of roof about 12 feet deep. It is deceptively hard to get out of, so take care if you drop down. Also note that the roof surface (which was replaced within the last few years) will rub off onto your clothing. Wear clothing that you don’t mind getting some white patches on (it washes out!). One section of the lower roof is a fantastic place to hang out, because it isn’t visible from any other building, and is sheltered from the wind. Be aware that some of the windows that allow access may be alarmed.
\subsubsection{West Upper Roof [5:D]}
Exit through the same windows as if you were headed to the lower roof, and instead get to the interior corners of the horseshoe-shaped upper roof. There you’ll find a “chimney” formed by the dormers (the protruding windows). Press your back into this corner and, using all 4 limbs, shimmy your way up the chimney and up onto the roof. Please be careful not to step on or grab onto the gutter on the edge of the roof, it will break! Additionally, if any of the slate tiles on the dormers come off, please move them out of the way of the roof access. As you climb onto the upper roof of West, know that you are silhouetted against the sky for any Public Safety officer making their patrol down Sage Ave. Stay low and freeze if you see Ford Explorer headlights. If they drive on by, no worries. If they pull into the parking lot, exit the roof ASAP. Note that the surface of the West roof will rub off on your clothes, so bring clothes that you don’t mind having some red patches on (it washes out!).

\section{Conclusion}

Thank you for reading the CEM and taking the time to make campus exploration a part of your student experience. We hope that if you’ve taken the time to read this far, you’ll take the time to read some really important notes about sustainable use.

Many places on campus have cool things, valuable things, or pristine spaces begging to be drawn on. There are in-tact windows, glass light fixtures, school signs and slogans and plenty of other things that you could deface in an instant. All we ask is that, if you use the CEM to guide your explorations on campus, you take sustainability seriously.
\subsection{Sustainability}

RPI’s student population numbers below 10,000, which is small for a research university in this day and age. We have a small campus with a lot of history. Many generations of students have and will want to experience the campus, including you. Don’t ruin it for other people. Urban explorers, and by the same token, campus explorers, have a tenuous but respected relationship with establishments. As long as you stay quiet, don’t break things and don’t deface publicly visible property, property owners and police are willing to patrol lightly. These rules hold true for campus property as well. 

Back in 2010, a student stole thousands of dollars worth of equipment from the basement of West, resulting in Public Safety patrols locking down the area, and multiple security measures being installed. Only since 2017, with these rooms being cleared out to be repurposed, have these spaces become accessible again by explorers. By the same token, after multiple graffiti problems on the North side of Sage Labs, security cameras were installed to watch the boiler room and surrounding grounds. 

When property owners, including RPI, are faced with activity that threatens their livelihood or public image, they’re forced to take action. As an explorer, don’t force RPI’s hand. RPI is an institution concerned with funding and attracting students. They are perfectly content to leave some areas accessible and allow students to draw on the walls and express themselves so long as it doesn’t impact that golden standard: money. 

In fact, RPI actually deliberately allows some “delinquency” to happen on campus for precisely this reason. If students cause “trouble” in areas where it really doesn’t affect the school, they are less likely to do it elsewhere. For your own sake as well as that of those like you, please let RPI continue to live within this assumption. Should the time come where you feel RPI has done something bad enough to warrant you spraypainting the DCC windows, no one will stop you. Until then, please keep the graffiti to low-traffic areas, don’t steal anything valuable, and please, please don’t go breaking things.
\subsection{Closing Remarks}
This guide has been the joint effort of multiple people working and exploring for many hours. Your input is valuable, and should you find information that isn’t in the CEM that you feel should be, please, let us know! Student exploration never has been and never should be a solo endeavor.

Moreover, the appendices that follow this document contain a wide variety of additional information. This documentation is less regulated and is far more open, but contains a wealth of knowledge if you know where to look. Please feel welcome to provide suggestions for appendix additions.

Keep exploring, friends, and we’re sure to see you out there.
\pagebreak

\section{Appendix}
\subsection{Appendix A: Disciplinary Information}
Following should go here
\begin{itemize}
  \item Relevant excerpts from student handbook
  \item People’s experiences with being caught by PubSafe
  \item Alan please add details
\end{itemize}
\subsection{Appendix B: Spots of Unique Visual Interest}
Looking to draw, paint, shoot some photos or take in a cool view for a moment? Check here for a list (in no particular order) of great spots on campus to indulge your inner artist.
% \setlength{\tabcolsep}{18pt}
\renewcommand{\arraystretch}{2}
\begin{center}
  \begin{tabular}{||p{7cm}|p{4cm}|p{4cm}||} 
    \hline
    Spot Description & Location & Best For \\
    % \hline
    \hline
    DCC Tunnel, a popular collaborative student art space. Notable for being painted over less often than the Throne Room
    & Tunnel connecting the DCC to the JEC, right next to WRPI office
    & Art and Photography\\
    \hline

    Carnegie Attic, A large, less-travelled attic located on the top floor of the Carnegie building. Plenty of blank space.
    &Behind the door at the top of the staircase of the Carnegie building.
    &Art and Photography.\\
    \hline

    “Throne Room 2.0,” another good graffiti location. Doesn’t appear to get painted over much, if at all.
    &Stairwell and landing that leads to the Sage West Attic.
    &Art\\
    \hline

    J-ROWL Tunnel, like the DCC tunnel but gets even less traffic. Lots of unused space.
    &Tunnel connecting the JEC to J-ROWL, accessible from either buildings’ basements.
    &Art and Photography\\
    \hline

    J-ROWL Tunnel Elevator, the white paneling of the interior and lack of use makes it a good place for large projects. 
    \textbf{Currently inaccessible}
    &On the J-ROWL side of the J-ROWL tunnel.
    &Art\\
    \hline

    The Throne Room, previously one of the best places for student art on campus. A collaborative space with plenty of room and lots of inspiration. \textbf{Risky as of fall 2019}
    &Top of the North-West CII stairwell, right next to the roof access.
    &Art and Photography\\
    \hline
  \end {tabular}
\end{center}

\subsection{Appendix C: Other Building Resources}

A complete list of floor plans for each floor of almost every building on campus (available as a .zip file): \href{http://zim2411.info/floorplans/}{http://zim2411.info/floorplans/}

\medskip
\noindent
The Folsom Library Archives, a dedicated historical archive of RPI: \href{http://archives.rpi.edu/}{http://archives.rpi.edu/}

\subsection{Appendix D: Steam Tunnels [3:F]}
The Elsewhere Club does not officially condone the exploration of the RPI steam tunnels. That being said, if you accept the risks of: asbestos inhalation, head-height climbing, squeezing through small spaces, and possible expulsion if caught, contact the owner of this document or moderator of Elsewhere for access to the steam tunnel map and guide.

\subsection{Appendix E: Further Exploration Around Troy}
For the purposes of exploring beyond the realm of RPI’s campus, the 1 to 6 System will still be used. The first number in any ranking ([x:x]) will still be physical risk, but the second number will now be legal risk as opposed to administrative risk. Off campus, Troy PD responds to calls, not Public Safety.

\subsubsection{Poesten Kill “The Gorge” [3:B]}
The Gorge is a beautiful place to explore year-round, with spectacular views of several waterfalls. In the spring, or after a period of high rainfall, the water in the Poesten Kill flows very fast, and every year there are injuries, and sometimes fatalities in the gorge. Always go with a group of people, and if you are going at night, bring flashlights. People often swim in the water of the Gorge, although there is likely agricultural runoff in the water from upstream as well as broken bottles or trash in certain spots near the water’s edge, so explorers should exercise caution. The wooded area of the Park can be a good spot to hunt for mushrooms.

Directions: from the RPI Union, walk south down 15th Street to Congress St. and take a right, and then a left, across the street down Cypress St. at the end of Cypress, there is a gated gravel road leading down to the gorge. Continue down this path until it becomes a trail leading down to the water. Take care on this path, as it becomes eroded throughout the year. 
If the water is very low, you can cross to Poesten Kill Gorge park.

While walking down the gravel road, to your left you will see a broken-down brick building with machinery visible inside of it. Climb the hill to get on top of the structure. There is an open hatch in the roof, and you can climb down inside using the vertical conveyor belt as a ladder. There are some underground pipes to crawl through. One goes horizontally and exits farther along the hill. The other slopes down and probably leads to Hell. 
\subsubsection{Troy Springs [1:A]}
A bit southeast and down the hill from the Poestenkill Gorge Park is a natural spring that has been diverted to run into spigots so that passers-by can collect the clean spring water that flows out. It’s popular with many Troy residents, who swear it’s some of the best drinking water around. The water seems to flow pretty consistently, even in winter. 

Directions: The spring is located in the small park at 58-78 Spring Avenue. From the Postenkill Gorge Park, take Linden Avenue down the hill, then take a left on Spring Avenue and keep going uphill until you see the concrete and pipe structure of the spring on the left.
\subsubsection{Upper Posten Kill [1:B]}
A small hidden path from the northern edge of Elmwood Hill Cemetery leads down to a beautifully secret section of the Posten Kill that rarely sees visitors. Bring a picnic and relax by the stream. Some spots are deep enough to swim, but take note of the current.

Directions: After entering Elmwood Cemetery through the front gate, proceed mournfully and respectfully along the main road and take the first left. At the first corner, take the small trail down the hill to the river.
\subsubsection{Posten Kill Dam and Facilities [2:C]}
Hudson Riverfront
\subsubsection{Hudson Riverfront (south-central) [1:A]}

Between Liberty Street and Adams Street in the north-south direction and between the Hudson River and Front Street in the west-east direction is a razed former industrial site with no barriers to entry and nobody watching it. The large concrete foundations of buildings that used to be here are still present, along with may have been some sort of concrete structure for removing goods from docked boats. Hardy opportunist tree and plant species can be seen growing out of the concrete.

\subsubsection{King Road [2:B]}
Directions: Proceed down Main St. in South Troy toward the Rensselaer County Prison. Take a right at the facility parking lot to get to the Burden Iron Works Museum which has a sizable collection of industrial-era bits and bobs and a bunch of rusty machinery outside, including a massive iron crucible, or a left down King Road. Along this road are a bunch of derelict buildings with interesting graffiti and urban exploration potential. Go in groups and don’t bring valuables, as some of the buildings may be inhabited.
\subsubsection{Historic Hydro Power Site [1:A]}
\subsubsection{Prospect Park Pool [2:B]}
\end{document}
